%  This is free software: you can redistribute it and/or modify
%  it under the terms of the GNU General Public License as published by
%  the Free Software Foundation, either version 3 of the License, or
%  (at your option) any later version.
%
%  This is distributed in the hope that it will be useful,
%  but WITHOUT ANY WARRANTY; without even the implied warranty of
%  MERCHANTABILITY or FITNESS FOR A PARTICULAR PURPOSE.  See the
%  GNU General Public License for more details.
%
%  You can find the GNU General Public License at <http://www.gnu.org/licenses/>.
\documentclass[a1paper,portrait,fontscale=0.54]{baposter}
\usepackage{tikz}
\usepackage{graphicx, svg}
\usepackage{amssymb}
\usepackage{mathtools}
\usepackage{multicol}
%%%%%%%%%%%%%%%%%%%%%%%%%%%%%%%%%%%%%%%%%%%%%%%%
% Language, Encoding and Fonts
% http://en.wikibooks.org/wiki/LaTeX/Internationalization
%%%%%%%%%%%%%%%%%%%%%%%%%%%%%%%%%%%%%%%%%%%%%%%%
% Select encoding of your inputs. Depends on
% your operating system and its default input
% encoding. Typically, you should use
%   Linux  : utf8 (most modern Linux distributions)
%            latin1
%   Windows: ansinew
%            latin1 (works in most cases)
%   Mac    : applemac
% Notice that you,fontscale=0.833 can manually change the input
% encoding of your files by selecting "save as"
% an select the desired input encoding.
\usepackage[utf8]{inputenc}
% Make latex understand and use the typographic
% rules of the language used in the document.
\usepackage[english]{babel}
% Use the vector font Latin Modern which is going
% to be the default font in latex in the future.
\usepackage{helvet}
% Change the default font family from roman to sans serif
\renewcommand{\familydefault}{\sfdefault} % for text
\usepackage[helvet]{sfmath} % for math
% Choose the font encoding
\usepackage[T1]{fontenc}

%%%%%%%%%%%%%%%%%%%%%%%%%%%%%%%%%%%%%%%%%%%%%%%%
% Graphics and Tables
% http://en.wikibooks.org/wiki/LaTeX/Importing_Graphics
% http://en.wikibooks.org/wiki/LaTeX/Tables
% http://pgfplots.sourceforge.net/
%%%%%%%%%%%%%%%%%%%%%%%%%%%%%%%%%%%%%%%%%%%%%%%%
% You cannot use floats in the baposter theme.
% We therefore load the caption package which provides
% the command \captionof
% Set up how figure and table captions are displayed
\usepackage{caption}
\captionsetup{
  font=small,% set font size to footnotesize
  labelfont=bf % bold label (e.g., Figure 3.2) font
}
% Make the standard latex tables look so much better
\usepackage{array,booktabs}
% For creating beautiful plots
\usepackage{pgfplots}

%%%%%%%%%%%%%%%%%%%%%%%%%%%%%%%%%%%%%%%%%%%%%%%%
% Mathematics
% http://en.wikibooks.org/wiki/LaTeX/Mathematics
%%%%%%%%%%%%%%%%%%%%%%%%%%%%%%%%%%%%%%%%%%%%%%%%
% Defines new environments such as equation,
% align and split
\usepackage{amsmath}
% Adds new math symbols
\usepackage{amssymb}

\DeclareMathOperator*{\argmax}{arg\,max}
\DeclareMathOperator*{\argmin}{arg\,min}
%%%%%%%%%%%%%%%%%%%%%%%%%%%%%%%%%%%%%%%%%%%%%%%%
% Colours
% http://en.wikibooks.org/wiki/LaTeX/Colors
%%%%%%%%%%%%%%%%%%%%%%%%%%%%%%%%%%%%%%%%%%%%%%%%
\selectcolormodel{RGB}
% define the three blue colors
\definecolor{nustblue}{RGB}{0,102,158}% dark blue
\definecolor{nustblue1}{RGB}{0, 189, 181} % light blue
\definecolor{nustblue2}{RGB}{0,153,237} % lighter blue
% \definecolor{nustblue3}{RGB}{100,100,255}
%%%%%%%%%%%%%%%%%%%%%%%%%%%%%%%%%%%%%%%%%%%%%%%%
% Lists
% http://en.wikibooks.org/wiki/LaTeX/List_Structures
%%%%%%%%%%%%%%%%%%%%%%%%%%%%%%%%%%%%%%%%%%%%%%%%
% Easier configuration of lists
\usepackage{enumitem}
%configure itemize
\setlist{%
  topsep=0pt,% set space before and after list
  noitemsep,% remove space between items
  labelindent=\parindent,% set the label indentation to the paragraph indentation
  leftmargin=*,% remove the left margin
  font=\color{nustblue}\normalfont, %set the colour of all bullets, numbers and descriptions to nustblue
}
% use set<itemize,enumerate,description> if you have an older latex distribution
\setitemize[1]{label={\raise1.25pt\hbox{$\blacktriangleright$}}}
\setitemize[2]{label={\scriptsize\raise1.25pt\hbox{$\blacktriangleright$}}}
\setitemize[3]{label={\raise1.25pt\hbox{$\star$}}}
\setitemize[4]{label={-}}
%\setenumerate[1]{label={\theenumi.}}
%\setenumerate[2]{label={(\theenumii)}}
%\setenumerate[3]{label={\theenumiii.}}
%\setenumerate[4]{label={\theenumiv.}}
%\setdescription{font=\color{nustblue}\normalfont\bfseries}

% use setlist[<itemize,enumerate,description>,<level>] if you have a newer latex distribution
%\setlist[itemize,1]{label={\raise1.25pt\hbox{$\blacktriangleright$}}}
%\setlist[itemize,2]{label={\scriptsize\raise1.25pt\hbox{$\blacktriangleright$}}}
%\setlist[itemize,3]{label={\raise1.25pt\hbox{$\star$}}}
%\setlist[itemize,4]{label={-}}
%\setlist[enumerate,1]{label={\theenumi.}}
%\setlist[enumerate,2]{label={(\theenumii)}}
%\setlist[enumerate,3]{label={\theenumiii.}}
%\setlist[enumerate,4]{label={\theenumiv.}}
%\setlist[description]{font=\color{nustblue}\normalfont\bfseries}

%%%%%%%%%%%%%%%%%%%%%%%%%%%%%%%%%%%%%%%%%%%%%%%%
% Misc
%%%%%%%%%%%%%%%%%%%%%%%%%%%%%%%%%%%%%%%%%%%%%%%%
% change/remove some names
\addto{\captionsenglish}{
  %remove the title of the bibliograhpy
  \renewcommand{\refname}{\vspace{-0.7em}}
  %change Figure to Fig. in figure captions
  \renewcommand{\figurename}{Fig.}
}
% create links
\usepackage{url}
%note that the hyperref package is currently incompatible with the baposter class

%%%%%%%%%%%%%%%%%%%%%%%%%%%%%%%%%%%%%%%%%%%%%%%%
% Macros
%%%%%%%%%%%%%%%%%%%%%%%%%%%%%%%%%%%%%%%%%%%%%%%%
\newcommand{\alert}[1]{{\color{nustblue}#1}}
\newcommand{\pro}{\includegraphics[scale=0.02]{images/pro.png}}
\newcommand{\con}{\includegraphics[scale=0.02]{images/con.png}}
\newcommand{\undecided}{\includegraphics[scale=0.004]{images/procon.png}}
%%%%%%%%%%%%%%%%%%%%%%%%%%%%%%%%%%%%%%%%%%%%%%%%
% Document Start
%%%%%%%%%%%%%%%%%%%%%%%%%%%%%%%%%%%%%%%%%%%%%%%%
\begin{document}
%%%%%%%%%%%%%%%%%%%%%%%%%%%%%%%%%%%%%%%%%%%%%%%%
% Some changes that cannot be made in the preamble
%%%%%%%%%%%%%%%%%%%%%%%%%%%%%%%%%%%%%%%%%%%%%%%%
% set the background of the poster
\background{
  \begin{tikzpicture}[remember picture,overlay]%
    %the poster background color
    \fill[fill=nustblue2] (current page.north west) rectangle (current page.south east);
    %the header
    \fill [fill=nustblue] (current page.north west) rectangle ([yshift=-\headerheight] current page.north east);
  \end{tikzpicture}
}
% if you want to reduce the space before and after equations, use and adjust
% the following lines
%\addtolength{\abovedisplayskip}{-2mm}
%\addtolength{\belowdisplayskip}{-2mm}

%%%%%%%%%%%%%%%%%%%%%%%%%%%%%%%%%%%%%%%%%%%%%%%%
% General poster setup
%%%%%%%%%%%%%%%%%%%%%%%%%%%%%%%%%%%%%%%%%%%%%%%%
\begin{poster}{
  %general options for the poster
  grid=false,
  columns=3,
%  colspacing=4.2mm,
  headerheight=0.1\textheight,
  background=none,
%  bgColorOne=red!42, %is used when background != user and none
%  bgColortwo=green!42, %is used when background is shaded
  eyecatcher=true,
  %posterbox options
  headerborder=closed,
  borderColor=nustblue,
  headershape=rectangle,
  headershade=plain,
  headerColorOne=nustblue,
%  headerColortwo=yellow!42, %is used when the header background is shaded
  textborder=rectangle,
  boxshade=plain,
  boxColorOne=white,
%  boxColorTwo=cyan!42,%is used when the text background is shaded
  headerFontColor=white,
  headerfont=\Large\sf\bf,
  linewidth=1pt
}
%the Eye Catcher (the logo on the left)
{
   %this can be commented out or replaced by a company/department logo
   % Logos UWr and KSI - one below the other
   \begin{minipage}[c]{0.15\textwidth}
     \centering
     \includegraphics[height=0.35\headerheight]{images/UWr_Logo.jpg} \\
     \vspace{0.2em}
     \includegraphics[height=0.32\headerheight]{images/KSI_Logo.png}
   \end{minipage}
   % Other logos
   % \hspace{0.3cm}
   % \begin{minipage}[c]{0.15\textwidth}
   %   \centering
   %   \includegraphics[height=0.28\headerheight]{images/Research_University_Horizontal.png}
   %   \vspace{0.2em}
   %  %  \hspace{0.3cm}
   %   \includegraphics[height=0.35\headerheight]{images/Arqus.png} \\
   % \end{minipage}
   % Old version (side by side) - commented out:
   % \includegraphics[height=0.35\headerheight]{images/UWr_Logo.jpg}
   % \hspace{.5cm}
   % \includegraphics[height=0.35\headerheight]{images/KSI_Logo.png}
}
%the poster title
{\color{nustblue}\bf
\Huge DINOmite: Adversarial Robustness \\of DINOv3 Vision Transformers % \\ \vspace{1mm}\Large {Subtitle}
}
%the author(s)
{\large
  \vspace{1em}
  \begin{minipage}[t]{0.18\textwidth}
    \centering
    Jan Burdzicki$^*$ \\
    {\small janburdzicki@gmail.com}
  \end{minipage}
  \hspace{0.2cm}
  \begin{minipage}[t]{0.18\textwidth}
    \centering
    Igor Jakus$^*$ \\
    {\small igorjakus@protonmail.com}
  \end{minipage}
  \hspace{0.2cm}
  \begin{minipage}[t]{0.18\textwidth}
    \centering
    Hubert Berlicki$^*$ \\
    {\small berlickihubert@gmail.com}
  \end{minipage}
  \hspace{0.2cm}
  \begin{minipage}[t]{0.25\textwidth}
    \centering
    Wojciech Aszkiełowicz$^*$ \\
    {\small w.aszkielowicz@proton.me}
  \end{minipage}
  \\[0.5em]
  Institute of Computer Science, University of Wrocław \\[0.3em]
  {\small $^*$Equal contribution}
}
%the logo (the logo on the right) - QR Code
{
  \vspace{0.5em}
  \includegraphics[height=0.7\headerheight]{images/qrcode.png}
}

%%%%%%%%%%%%%%%%%%%%%%%%%%%%%%%%%%%%%%%%%%%%%%%%
% the actual content of the poster begins here
%%%%%%%%%%%%%%%%%%%%%%%%%%%%%%%%%%%%%%%%%%%%%%%%

% ============================================================================
% COLUMN 1
% ============================================================================

\begin{posterbox}[name=abstract, column=0, row=0]{Abstract}
\textbf{Problem:} DINOv3 Vision Transformers achieve high accuracy but are vulnerable to adversarial attacks.

\textbf{Contribution:} Comprehensive evaluation framework testing 5 attack methods (FGSM, PGD, BIM, C\&W, DeepFool) and 3 defense strategies (PGD-AT, TRADES, MART).

\textbf{Key Results:}
\begin{itemize}
    \item Standard models: 85\% clean, 25\% robust (PGD)
    \item With defense: 82\% clean, 60\% robust (TRADES)
    \item Trade-off manageable with proper training
\end{itemize}
\end{posterbox}

\begin{posterbox}[name=motivation, column=0, below=abstract]{Motivation}
\subsection*{Why It Matters}
\begin{itemize}
    \item Safety-critical: autonomous vehicles, medical diagnosis
    \item Real-world threats: adversarial examples work in physical world
    \item Trust in AI: robust models essential for deployment
\end{itemize}

\subsection*{Research Questions}
\begin{enumerate}
    \item How robust are DINOv3 models?
    \item Which defenses work best?
    \item Accuracy vs robustness trade-off?
\end{enumerate}
\end{posterbox}

\begin{posterbox}[name=problem, column=0, below=motivation]{Problem Definition}
\subsection*{Adversarial Examples}
$\min_{\delta} \|\delta\|_p$ s.t. $f(x + \delta) \neq f(x)$, $\|\delta\|_\infty \leq \epsilon$

\subsection*{Threat Model}
\begin{itemize}
    \item White-box: full model access
    \item L$\infty$ constraint: $\epsilon = 8/255$
    \item Untargeted attacks
\end{itemize}

\subsection*{Metrics}
\begin{itemize}
    \item Clean Accuracy
    \item Robust Accuracy
    \item Attack Success Rate
\end{itemize}
\end{posterbox}

% ============================================================================
% COLUMN 2
% ============================================================================

\begin{posterbox}[name=attacks, column=1, row=0]{Attack Methods}
\subsection*{FGSM}
(Goodfellow et al., 2015)\\
Single-step: $x_{adv} = x + \epsilon \cdot \text{sign}(\nabla_x L)$

\subsection*{PGD}
(Madry et al., 2018)\\
Iterative FGSM with projection. \textbf{Strongest attack.}

\subsection*{BIM}
(Kurakin et al., 2017)\\
 Iterative FGSM, smaller steps

\subsection*{C\&W}
(Carlini \& Wagner, 2017)\\
 Optimization-based, L2 norm

\subsection*{DeepFool}
(Moosavi-Dezfooli et al., 2016)\\
 Minimal perturbation to cross boundary
\end{posterbox}

\begin{posterbox}[name=defenses, column=1, below=attacks]{Defense Methods}
\subsection*{PGD-AT}
(Madry et al., 2018)\\
 Train on PGD: $\min_\theta \mathbb{E}[\max_{\delta} L(f(x+\delta), y)]$

\subsection*{TRADES}
(Zhang et al., 2019)\\
Trade-off: $L_{nat} + \beta \cdot KL(p_{adv} \| p_{nat})$

\subsection*{MART}
(Wang et al., 2020)\\
Focus on misclassified examples
\end{posterbox}

\begin{posterbox}[name=examples, column=1, below=defenses]{Adversarial Examples}
% PLACEHOLDER: Add images from adversarial_examples/
% Uncomment and adjust paths:
% \includegraphics[width=0.95\textwidth]{../adversarial_examples/pgd_example_cifar10_1.png}
\begin{center}
\textit{[PLACEHOLDER: Adversarial example images]}
\end{center}

\textit{Example: Original (left) vs PGD attack (right)}
\end{posterbox}

% ============================================================================
% COLUMN 3
% ============================================================================

\begin{posterbox}[name=experiments, column=2, row=0]{Experimental Setup}
\subsection*{Model}
\begin{itemize}
    \item DINOv3 ViT-S/16 (pretrained)
    \item Linear head: 384 $\rightarrow$ num\_classes
    \item Frozen backbone, trainable head
\end{itemize}

\subsection*{Datasets}
\begin{itemize}
    \item CIFAR-10 (primary)
    \item GTSRB (safety-critical)
    \item Tiny ImageNet
\end{itemize}

\subsection*{Evaluation}
\begin{itemize}
    \item 1000 test samples
    \item $\epsilon \in \{0, 1/255, 2/255, 4/255, 8/255, 16/255\}$
    \item Multiple seeds
\end{itemize}
\end{posterbox}

\begin{posterbox}[name=results, column=2, below=experiments]{Results}
% PLACEHOLDER: Add results table
% Use: \input{results_table.tex} or manually create table
\begin{center}
% Uncomment to add table:
% \begin{tabular}{lccc}
% \toprule
% Model & Clean & FGSM & PGD \\
% \midrule
% Original & 85.0 & 45.0 & 25.0 \\
% PGD-AT & 82.0 & 65.0 & 55.0 \\
% \bottomrule
% \end{tabular}
\textit{[PLACEHOLDER: Results table]}
\end{center}

\subsection*{Key Findings}
\begin{enumerate}
    \item Standard models: vulnerable (25\% robust)
    \item PGD strongest attack
    \item Defenses help: 60\% robust (TRADES)
    \item Trade-off manageable
\end{enumerate}
\end{posterbox}

\begin{posterbox}[name=curves, column=2, below=results]{Robustness Curves}
% PLACEHOLDER: Add robustness curve plot
% Uncomment and adjust path:
% \includegraphics[width=0.95\textwidth]{../results/robustness_comparison_pgd.png}
\begin{center}
\textit{[PLACEHOLDER: Plot: Accuracy vs Epsilon]}
\end{center}

\textit{Shows how accuracy drops with increasing attack strength}
\end{posterbox}

% ============================================================================
% BOTTOM SECTION (SPANS ALL COLUMNS)
% ============================================================================

\begin{posterbox}[name=comparison, column=0, below=problem, span=3]{Model Comparison}
% PLACEHOLDER: Add comparison bar chart
% Uncomment and adjust path:
% \includegraphics[width=0.95\textwidth]{../results/attack_comparison_original.png}
\begin{center}
\textit{[PLACEHOLDER: Bar chart comparing models]}
\end{center}

\begin{itemize}
    \item \textbf{Original:} High clean (85\%), low robust (25\%)
    \item \textbf{PGD-AT:} Balanced (82\% / 55\%)
    \item \textbf{TRADES:} Best robustness (81.5\% / 60\%)
    \item \textbf{MART:} Good balance (81.8\% / 58\%)
\end{itemize}
\end{posterbox}

\begin{posterbox}[name=conclusions, column=0, below=comparison, span=3]{Conclusions \& Future Work}
\subsection*{Main Takeaways}
\begin{itemize}
    \item DINOv3 vulnerable without defense
    \item Adversarial training effective (60\% robust)
    \item Trade-off manageable
    \item Framework ready for deployment
\end{itemize}

\subsection*{Future Work}
\begin{itemize}
    \item More datasets (GTSRB, Tiny ImageNet full)
    \item Certified defenses
    \item Ensemble methods
    \item Real-world testing
\end{itemize}
\end{posterbox}

\begin{posterbox}[name=references, column=0, below=conclusions, span=3]{References}
\begin{multicols}{3}
\begin{enumerate}
    \item Goodfellow et al. (2015). \textit{arXiv:1412.6572}\\
    \url{https://arxiv.org/abs/1412.6572}
    \item Moosavi-Dezfooli et al. (2016). \textit{arXiv:1511.04599}\\
    \url{https://arxiv.org/abs/1511.04599}
    \item Kurakin et al. (2017). \textit{arXiv:1607.02533}\\
    \url{https://arxiv.org/abs/1607.02533}
    \item Carlini \& Wagner (2017). \textit{arXiv:1608.04644}\\
    \url{https://arxiv.org/abs/1608.04644}
    \item Madry et al. (2018). \textit{arXiv:1706.06083}\\
    \url{https://arxiv.org/abs/1706.06083}
    \item Zhang et al. (2019). \textit{arXiv:1901.08573}\\
    \url{https://arxiv.org/abs/1901.08573}
    \item Wang et al. (2020). \textit{OpenReview}\\
    \url{https://openreview.net/pdf?id=rklOg6EFwS}
\end{enumerate}
\end{multicols}
\end{posterbox}


\end{poster}
\end{document}
